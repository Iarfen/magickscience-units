SCIFIR UNITS -\/ VERSION 2

// NUMBER OF TODOS\+: 52

// MATERIAL\+\_\+\+VECTOR\+\_\+\+UNIT (1) // TODO\+: add operator== and operator!=

// TENSORS (2) // TODO\+: add display functions without new lines and other equivalents with new lines, they should allow to display like the other units related to dimensions // TODO\+: add operator$\ast$, \mbox{\hyperlink{namespacescifir_ac62851b9d931d02184edca7f3db3f1a7}{cross\+\_\+product()}}, \mbox{\hyperlink{namespacescifir_a732fc05956fd18f78cd0019c5c023b5c}{dot\+\_\+product()}}, hadamard\+\_\+product()

// TRAJECTORY AND PARAMETERIZATION // TODO\+: think if to add here the trajectory and the parameterization

// BUGS AND SIMILAR PROBLEMS (3) // TODO\+: add a whitespace between operator and the = sign to all those functions

// TODO\+: update all the uses of grade in a function name or any other part by degree, which is a more precise word // TODO\+: percentage should have an enum percentage\+::type with the values PERCENTAGE, PARTS\+\_\+\+PER\+\_\+\+MILLION, PARTS\+\_\+\+PER\+\_\+\+BILLION, PARTS\+\_\+\+PER\+\_\+\+TRILLION, PARTS\+\_\+\+PER\+\_\+\+QUATRILLION // TODO\+: document the light units somewhere

// TODO\+: complex\+\_\+number$<$$>$ and lab\+\_\+number$<$$>$ should work with scalar\+\_\+unit class as template parameter too, not only with their derived classes

// TODO\+: custom dimensions should allow to store their names too, not only their symbols // TODO\+: maybe the prefix should be also present in the operator == of dimensions, and then it\textquotesingle{}s needed another function to do what this operator is currently doing

// TODO\+: \mbox{\hyperlink{namespacescifir_a76943d4b7ab2e9ba278182abafa5378a}{display\+\_\+double()}} and \mbox{\hyperlink{namespacescifir_ab22ad226e1213d2c21d0f62848c03b39}{display\+\_\+long\+\_\+double()}} functions of \mbox{\hyperlink{types_8cpp}{types.\+cpp}}

// TODO\+: create percentage$<$double$>$ and percentage$<$long double$>$ with ppb, ppt and ppq. Create also the function is\+\_\+long\+\_\+percentage() to test for ppb, ppt and ppq

// BUGS AND SIMILAR PROBLEMS -\/ HARD (4) // TODO\+: check and correct all the warnings of cl.\+exe with /\+Wall // TODO\+: -\/Wall gives a warning for vector\+\_\+unit\+\_\+nd with end of non-\/void function // TODO\+: look at the 1.\+99 of test\+\_\+coordinates\+\_\+2d vs the 2 of the float specialization // TODO\+: check the literal of Ohm initialization in string class to be used instead of IS\+\_\+\+UNIX and IS\+\_\+\+WINDOWS in all the code files. Also, possibly it\textquotesingle{}s needed to add IS\+\_\+\+UNIX and IS\+\_\+\+WINDOWS there

// BUILD (3) // TODO\+: support freebsd and openbsd // TODO\+: upload the package to the container registry of github packages // TODO\+: search if there\textquotesingle{}s a way to add multiple jobs to the build with the command cmake --build, similar to make -\/j

// BUILD -\/ ADVANCED (7) // TODO\+: study bazel and look the files of bazel that are part of Catch2, maybe the same should be done for scifir-\/units // TODO\+: see how to support consoles inside the building, if possible // TODO\+: possibly add the library to conan // TODO\+: test vcpkg.\+json uploading scifir-\/units to vcpkg // TODO\+: check the welcome file for cpack, maybe it\textquotesingle{}s useful // TODO\+: support the compilation with Emscripten and add web-\/assembly as another preset more // TODO\+: check the configuration options that exist for gcov

// IDE (2) // TODO\+: check the extension gcov viewer is working properly // TODO\+: configure ctest inside Visual Studio Code

// GITHUB (8) // TODO\+: update the github action that updates the \mbox{\hyperlink{CHANGELOG_8md}{CHANGELOG.\+md}} file, make it work // TODO\+: release the version 2.\+0-\/beta // TODO\+: add the github action with windows // TODO\+: add the github action with macos // TODO\+: add try online badge // TODO\+: add the release badge // TODO\+: add the code coverage badge // TODO\+: upload scifir-\/units to sourceforge too

// TESTS (5) // TODO\+: finish the tests of all classes // TODO\+: review get\+\_\+nd() of the tests of vector\+\_\+unit\+\_\+nd and coordinates\+\_\+nd // TODO\+: verify that are present all functions of coordinates with rotations // TODO\+: report the bug to CATCH of the use of bool() // TODO\+: verify the titles of sections of tests of units, meca\+\_\+number and special\+\_\+units // TODO\+: test of unit basic classes

// DOCUMENTATION (23) // TODO\+: change in the master branch the Doxyfile in order to only generate HTML, in order to improve performance // TODO\+: configure to update gh-\/pages based on some factor, maybe a pull request // TODO\+: create all the dox files of the library // TODO\+: document the thinking behind the geographic coordinates of allowing to change planet. By default it\textquotesingle{}s the Earth // TOOD\+: document that angle always uses grades // TODO\+: document the use of special characters in prefixes // TODO\+: document the special characters in vector units // TODO\+: document the special character in angle class // TODO\+: document what is a dimension, what is a derived dimension, and what is a basic dimension // TODO\+: use the bold for \char`\"{}internal function\char`\"{} inside the documentation // TODO\+: add the abbreviations in the groups too // TODO\+: document the initialization string of each class in their respective page // TODO\+: add latex to the factors of prefixes in \mbox{\hyperlink{prefix_8dox}{prefix.\+dox}} // TODO\+: document how the england and united states units work, which are different than the SI system of units

// TODO\+: document the coordinates for the universe, publish that too as an standard inside scifir // TODO\+: check what to do with the libraries of ISOs and document them inside the \mbox{\hyperlink{README_8md}{README.\+md}} file. If there isn\textquotesingle{}t a library of ISOs, decide if to create one // TODO\+: reference books related to the physics, chemistry, biology or math implemented // TODO\+: explain the custom dimensions in the file \mbox{\hyperlink{README_8md}{README.\+md}} // TODO\+: display as plain text the $<$\+T$>$ inside the md file // TODO\+: document the demonstration of why a number or a scalar unit divided by a vector should give a vector, explaining that it\textquotesingle{}s the reverse as the multiplication

// TODO\+: document the use of the custom\+\_\+display \char`\"{}sci\char`\"{} // TODO\+: think on the different cases of use of coordinates and document them, including the use of an origin // TODO\+: mention the WGS in the part of coordinates // TODO\+: document the use of fabs() and display() to compare values of units // TODO\+: document the special initialization of mass and mole with percentage // TODO\+: add initialization of lab\+\_\+number$<$$>$ inside \mbox{\hyperlink{README_8md}{README.\+md}} file

// TODO\+: add the summary sheet somewhere // TODO\+: \char`\"{}\+Things to remember\char`\"{} section, which explains the use of the degree for vector\+\_\+unit\+\_\+nd instead of the other two cases, the use of fabs, the use of display that doesn\textquotesingle{}t displays all decimals // TODO\+: add a section of \char`\"{}\+Optional knowledge\char`\"{}

// TODO\+: formulas of physics related to units // TODO\+: formulas of chemistry related to units // TODO\+: formulas of biology related to units // TODO\+: document how all the operations of scalars and vector units work mathematically, possibly use mathjax inside markdown to display formulas

// TODO\+: think if to add the help of Eclipse, and search if other IDEs use some help too // TODO\+: decide if to generate documentation for man, and if so, explain that inside the \mbox{\hyperlink{README_8md}{README.\+md}} file // TODO\+: document all the NEWS file

// TODO\+: write an specification of scifir-\/units to be implemented in any programming language. Write there too the generic method for writing bindings which is to maintain all function names and classnames equal, changing only their spelling to snake-\/case or camel-\/case as used in the programming language, and also document equivalencies like using static methods in Java inside a class instead of normal functions in C++.

// COORDINATES AND POINTS (5) // TODO\+: add coordinates\+\_\+3d\+::origin() to allow to initialize with an origin (another coordinates class of the same type), it\textquotesingle{}s specially useful for geographic coordinates

// TODO\+: the case when the initialization string contains only the character \textquotesingle{},\textquotesingle{}, without a value, has to initialize to zero

// TODO\+: display functions of coordinates and point classes should allow to change dimensions for any case, with another function identical in name but with the argument of the string of dimensions // TODO\+: all display functions of coordinates classes need an int argument to select number of decimals // TODO\+: is\+\_\+coordinates\+\_\+1d(), is\+\_\+coordinates\+\_\+2d(), etc

// COORDINATES AND POINTS -\/ FINISH GEOGRAPHIC COORDINATES (3) // TODO\+: finish get\+\_\+altitude(). Read about geographic coordinates and decide what to do in point\+\_\+3d, point\+\_\+nd, coordinates\+\_\+3d and coordinates\+\_\+nd // TODO\+: add the construction with an string for geographic coordinates // TODO\+: study the geographic coordinates deeply, and see if to add something more related to them

// VECTOR FIELDS (2) // TODO\+: vector\+\_\+field (it should operate with vector\+\_\+units and maybe with scalar\+\_\+units. With (x,y,z) it gives the respective vector). +, -\/ and $^\wedge$ with vector\+\_\+field // TODO\+: parsing of initialization strings of fields. Allow to declare variables that can be refered by name inside initialization strings of vector fields

// PLOTTING // TODO\+: add functions that allow to use also matplot++ inside the library

// ALGEBRA (4) // TODO\+: constructor with initialization string // TODO\+: is\+\_\+algebraic\+\_\+expression() and is\+\_\+algebraic\+\_\+term() functions // TODO\+: static variable storing scalar\+\_\+units, scalar\+\_\+fields, etc // TODO\+: search if there\textquotesingle{}s a library that displays equations inside the command-\/line with math format, with math glyphs // TODO\+: to\+\_\+latex() function that gives the algebraic\+\_\+expression in latex format

// C++ (6) // TODO\+: use constexpr and consteval to initialize units instantly, possibly use it in more parts of the code // TODO\+: change the use of const-\/reference to value and std\+::move in constructors // TODO\+: change the use of const-\/reference to value and std\+::move in other places // TODO\+: use reference when instantiating variables and when it\textquotesingle{}s not needed to instantiate by value inside the body of functions // TODO\+: if size\+\_\+t and other similar constants are intended to be used inside normal code // TODO\+: replace boost\+::split by split of C++20

// DIMENSIONS (7) // TODO\+: support the binary prefixes too // TODO\+: a new function to display dimensions with their full name, including the prefix // TODO\+: function get\+\_\+frequent\+\_\+dimensions(), which are the frequent definitions (J is N$\ast$m, W is J/s, etc) // TODO\+: functions get\+\_\+plural(), get\+\_\+fullname() and get\+\_\+fullplural()

// TODO\+: allow to initalize grade dimensions by symbol // TODO\+: angle class needs a to\+\_\+scalar\+\_\+unit() function that gives an equivalent scalar\+\_\+unit class with the same value and grade dimension // TODO\+: some dimensions like grade, rad and money shouldn\textquotesingle{}t be allowed to have a prefix // TODO\+: check the literal with e // TODO\+: read the ISO 80000

// TODO\+: finish the test of custom\+\_\+basic dimensions and document them in the \mbox{\hyperlink{README_8md}{README.\+md}} file

// UNITS (9) // TODO\+: solve in some way the problem that vector\+\_\+unit\+\_\+3d needs different characters in Windows an inside Linux when initialized with strings // TODO\+: allow to display in any conversion. By default it should always display in SI units, only if a conversion is expressly specified in the display functions the conversion is then the dimension that gets displayed // TODO\+: check dimensions in all inheriting classes of scalar\+\_\+unit and vector\+\_\+unit, it\textquotesingle{}s needed another constructor that checks them // TODO\+: vector\+\_\+unit\+\_\+3d class maybe need the comparison operators with themselves // TODO\+: use the PI of the std library // TODO\+: function point\+\_\+to() to a point and point\+\_\+to() to a coordinate // TODO\+: support the brackets inside \mbox{\hyperlink{namespacescifir_a35eaaedc7d36d386d328c0422e08863e}{is\+\_\+scalar\+\_\+unit()}}, \mbox{\hyperlink{namespacescifir_aa28fb8e3c6aa92108db7faac70b47f61}{is\+\_\+complex()}} and \mbox{\hyperlink{namespacescifir_a6f938e487f453f713e335dd6f884a413}{is\+\_\+lab\+\_\+number()}} // TODO\+: possibly add the light\+\_\+scalar\+\_\+unit class that allows operations with scalar\+\_\+unit classes, which must contain only one dimension, not a vector$<$dimension$>$ // TODO\+: function to\+\_\+latex() for dimensions and scalar\+\_\+unit

// TODO\+: Regex that checks all the invalid dimensions initialization inside a static\+\_\+assert (create a static function of valid\+\_\+initialization\+\_\+string()). Maybe try first by undefining the value if there\textquotesingle{}s something that doesn\textquotesingle{}t exist (with an else). It\textquotesingle{}s possible to test, with static\+\_\+assert, that dimension == nullptr, abbreviation == nullptr and conversion == nullptr // TODO\+: Detect when there\textquotesingle{}s the same dimension at the numerator and at the denominator of the string initialization

// TODO\+: make scalar\+\_\+unit a template class converting the default type of the value member-\/variable to float type. Change the derived units to template classes too, and also all vector\+\_\+units. Change the macro that defines derived units to be only the macro with HPP and use it in all predefined\+\_\+unit files

// PREDEFINED UNITS (4) // TODO\+: think if to add accoustic and matter predefined units, or if it\textquotesingle{}s not needed // TODO\+: delete all field classes that currently are vector\+\_\+units, and make them fields // TODO\+: think if to add another concentration class, the previous one has been deprecated // TODO\+: move cas\+\_\+number to the library of scifir-\/info, or even to another category

// MECA NUMBERS (2) // TODO\+: add the allowed typenames to lab\+\_\+number, and don\textquotesingle{}t accept any other type // TODO\+: \+\_\+angle, \+\_\+grade and \+\_\+radian literals // TODO\+: angle should read initialization strings in radian too, maybe with \char`\"{}radian\char`\"{} name, possibly \char`\"{}rad\char`\"{} should be supported too // TODO\+: add obtusangle, rect angle, acutangle, etc

// SPECIAL UNITS -\/ EXTRA (7) // TODO\+: ip class? check networking libraries of C++ and decide if to add it here // TODO\+: nutrition\+\_\+information // TODO\+: class for geographical position including ZID and coordinates\+\_\+3d // TODO\+: maybe pixel should be called pixel\+\_\+length // TODO\+: think if to add a default case for the none value of aid and for the none value of zid // TODO\+: functions is\+\_\+aid() and is\+\_\+zid() // TODO\+: initialize\+\_\+from\+\_\+string() for pH and p\+OH classes should work with pH and p\+OH at the start, respectively // TODO\+: maybe add zoom class, which allows to work with zooms // TODO\+: \+\_\+percentage literal

// EMOTIONAL UNITS (1) // TODO\+: finish the enums of \mbox{\hyperlink{mind_8hpp}{mind.\+hpp}}

// SPECIAL UNITS (4) // TODO\+: complete color class like coordinate classes, with all the getters of all the different color versions, like get\+\_\+h(), get\+\_\+s(), get\+\_\+v() // TODO\+: pixel\+\_\+color$<$$>$. Use monochrome\+\_\+pixel, truecolor\+\_\+pixel, etc, as typedefs of pixel\+\_\+color$<$$>$ // TODO\+: complex\+\_\+number$<$$>$ should have trigonometric functions for complex numbers // TODO\+: maybe create a mesh\+\_\+3d class, or vector$<$point\+\_\+3d$<$$>$$>$

// PREDEFINED PHYSICS UNITS (4) // TODO\+: electric\+\_\+field which calculates based on coulomb charges // TODO\+: gravity\+\_\+field // TODO\+: possibly magnetic\+\_\+field? // TODO\+: electric\+\_\+current?

// CONTROL VOLUME (1) // TODO\+: think what to do with the control\+\_\+volume

// UNITS -\/ ADVANCED (9) // TODO\+: \mbox{\hyperlink{namespacescifir_adeff4b825414d35ce963977f181f1b5c}{sqrt()}} and \mbox{\hyperlink{namespacescifir_a44533727ef4f03e8303664cdd665c2bc}{pow()}} maybe should be direct for created units, instead of passing by scalar\+\_\+unit again to initialize after that the other unit // TODO\+: check if it\textquotesingle{}s needed to add a function is\+\_\+si\+\_\+basic\+\_\+dimension() that gives whether the dimension is basic or not in the sense of the SI system of units // TODO\+: support and UTF32 string constructor for scalar\+\_\+unit in order to allow to create dimensions directly with some Unicode characters that are not present in UTF8 // TODO\+: scalar\+\_\+unit should have is\+\_\+valid() with some system // TODO\+: support the conversions with constexpr // TODO\+: add the operators +,-\/,$\ast$ and / in the derived classes of scalar\+\_\+unit and vector\+\_\+unit with the same class in order to avoid to check that the dimensions are the same, that saves time // TODO\+: check the object code resulting by testing different functions of the unit classes // TODO\+: 2d display of scalar\+\_\+units and of vector\+\_\+units (create a scifir\+\_\+units\+\_\+2d library for it) // TODO\+: 3d display of scalar\+\_\+units and of vector\+\_\+units (create a scifir\+\_\+units\+\_\+3d library for it)

// UNITS -\/ ADVANCED -\/ REDUCTION OF MEMORY CONSUMPTION (2) // OPTION 1\+: maybe delete the dimensions member-\/variable of scalar\+\_\+unit, and use instead another system for handling prefixes. The dimensions can be automatic based on their class. One possibility is to use an empty array and, when it\textquotesingle{}s empty, to send the fixed dimensions of the class instead, and only when changing something to add the dimensions there // OPTION 2\+: divide single dimensions unit of multiple-\/dimensions unit by adding only one dimension instead of the vector$<$dimension$>$ // OPTION 3\+: maybe the prefix and the dimension can be removed as member-\/variables if displaying automatically in some way or another, as it\textquotesingle{}s expressly specified. That is maybe the biggest optimization possible // OPTION 4\+: light\+\_\+length which uses only an enum of prefixes and a value, maybe it should be called length, and length should be called full\+\_\+length // OPTION 4 -\/ TODO\+: add const to the enum of light\+\_\+unit // AFTER SOME OPTION -\/ TODO\+: finish initial\+\_\+dimensions\+\_\+get\+\_\+structure() and get\+\_\+dimensions\+\_\+match() related to the new implementation

// ANOTHER PROJECT -\/ LIBRARY OF INFORMATION // TODO\+: isbn class // TODO\+: issn class

// ISOs // TODO\+: Publish the ISO of geographic location based on aid and zid classes // TODO\+: See if to make an ISO of an official symbol for money (not a concrete money of a country, but a universal one) // TODO\+: Add \char`\"{}depth\char`\"{} to an ISO of names for the lengths of objects (width, height and depth are the names). It\textquotesingle{}s needed to have a name in spanish for the depth too // TODO\+: Maybe create an \char`\"{}\+ISO\char`\"{} of geographic positioning taking the major axis of the planet, which can be the Earth or not, and adding 50 km to it, in order to have a border of safety in order to be sure that no point remains uncovered by the imaginary sphere that the geographic positioning creates around the planet. It can be used for any planet of the universe. The center of the planet is considered always the geometrical one, not the center of mass, because that last one changes with changes of the distribution of mass inside the planet

// ISO C++ // TODO\+: add º to the string literals // TODO\+: add \% to the string literals // TODO\+: add the possibility to create class names starting with numbers

// ELECTRONICS // TODO\+: check sensor libraries and decide which ones to support inside scifir-\/units (maybe in a new scifir library if needed)

// PATTERNS // TODO\+: possibly implement a pattern class using a regular expressions library

// EXTRA TOOLS // TODO\+: Create scicalcs, a cli tool that calculates with scifir-\/units any value

// PORTS // TODO\+: Port to C\# // TODO\+: Port to Java // TODO\+: Port to Octave // TODO\+: Port to Visual Basic

// TESTS // TODO\+: test of sizeof for all unit classes // TODO\+: benchmark test for scalar\+\_\+unit, comparing them to a float

// DOCUMENTATION (9) // TODO\+: document the point of view of the library of when a dimension is considered \char`\"{}basic\char`\"{} // TODO\+: document the ISOs important to use with this library // TODO\+: document a little how to handle currency // TODO\+: document that the pixel in dimension is only as length, not as a pixel on the screen as is in the pixel class // TODO\+: document how ppm and ppb work, also in the theorical sense // TODO\+: document an example of converting all currencies to money dimension, with different values. Use the currency abreviations of the ISO of currencies // TODO\+: add nomenclature of units // TODO\+: think if to add the functions of calculations or to add example of calculations in the documentation // TODO\+: document the explanation of what each unit means, given the defintion of the SI or of the entity that corresponds to reference

// RELEASE (3) // TODO\+: configure CMake with cpack // TODO\+: see what to do to configure optimizations // TODO\+: add scifir-\/units to the official repository of vcpkg

// MATRIX // TODO\+: See if it\textquotesingle{}s best to use template arguments for row and column or if to store those values as member-\/variables // TODO\+: Multiplication of matrices of different but compatible types // TODO\+: typecast to other matrix-\/classes of important libraries // TODO\+: Iterator with range to use only one range-\/for // TODO\+: Check limits of matrices for all operators // TODO\+: Use the GSL to implement the reverse matrix

// CONSTANTS // TODO\+: make a list of all important constants of science, with their respective unit. The constants of physics, chemistry and biology should be inside. Also, add the constants of astronomy

// FUTURE // TODO\+: support the case of n dimensions fixed // TODO\+: add the theta and phi characters to C++ variable names, and add them then to the member-\/variables of vector\+\_\+unit classes, and any other case of similar use. Add the symbol º to string literals // TODO\+: add the astronomy coordinates // TODO\+: add the other orthogonal coordinates, like paraboloidal // TODO\+: add to the ISO of the keyboards some system to write pi, theta, phi, among other symbols, with the keyboard in an easy way, without having to memorize any numeric code // TODO\+: propose an ISO symbol for money in general? // TODO\+: finish the empty array implementation for dimension, in order to have normal dimensions, no custom dimensions, of size 3 instead of size 6

// FUTURE -\/ MECA NUMBERS (POSSIBLE, THINK) // TODO\+: Add names to the meca numbers (angler, laber, etc) // TODO\+: The interval number class // TODO\+: The interval number subclasses of other numbers // TODO\+: The bounce number class // TODO\+: The percentage number class (it has to have the calculate function in order to receive a value to be the percentage of) // TODO\+: Solve the problem with left and right repeated (it\textquotesingle{}s not exclusive for direction\+\_\+symbol) // TODO\+: solid\+\_\+angle class (maybe it isn\textquotesingle{}t a meca number) // TODO\+: maybe \+\_\+angle for angle in order to use cos(x),sin(x),etc with degrees

// READINGS // Unit of measurement\+: \href{https://en.wikipedia.org/wiki/Unit_of_measurement}{\texttt{ https\+://en.\+wikipedia.\+org/wiki/\+Unit\+\_\+of\+\_\+measurement}} // International system of units\+: \href{https://en.wikipedia.org/wiki/International_System_of_Units}{\texttt{ https\+://en.\+wikipedia.\+org/wiki/\+International\+\_\+\+System\+\_\+of\+\_\+\+Units}} // Angle\+: \href{https://en.wikipedia.org/wiki/Angle}{\texttt{ https\+://en.\+wikipedia.\+org/wiki/\+Angle}} // Metrology\+: \href{https://en.wikipedia.org/wiki/Metrology}{\texttt{ https\+://en.\+wikipedia.\+org/wiki/\+Metrology}} // Color\+: \href{https://en.wikipedia.org/wiki/Color}{\texttt{ https\+://en.\+wikipedia.\+org/wiki/\+Color}} // RGB color model\+: \href{https://en.wikipedia.org/wiki/RGB_color_model}{\texttt{ https\+://en.\+wikipedia.\+org/wiki/\+RGB\+\_\+color\+\_\+model}} // Color model\+: \href{https://en.wikipedia.org/wiki/Color_model}{\texttt{ https\+://en.\+wikipedia.\+org/wiki/\+Color\+\_\+model}} // Unit prefix\+: \href{https://en.wikipedia.org/wiki/Unit_prefix}{\texttt{ https\+://en.\+wikipedia.\+org/wiki/\+Unit\+\_\+prefix}} // Metric prefix\+: \href{https://en.wikipedia.org/wiki/Metric_prefix}{\texttt{ https\+://en.\+wikipedia.\+org/wiki/\+Metric\+\_\+prefix}} 